% This file was converted to LaTeX by Writer2LaTeX ver. 1.0.2
% see http://writer2latex.sourceforge.net for more info
\documentclass[twoside,letterpaper]{article}
\usepackage[latin1]{inputenc}
\usepackage[T1]{fontenc}
\usepackage[english]{babel}
\usepackage{amsmath}
\usepackage{amssymb,amsfonts,textcomp}
\usepackage{color}
\usepackage{array}
\usepackage{supertabular}
\usepackage{hhline}
\usepackage{hyperref}
\hypersetup{pdftex, colorlinks=true, linkcolor=blue, citecolor=blue, filecolor=blue, urlcolor=blue, pdftitle=SYSTEM AND SOFTWARE ARCHITECTURAL AND DETAILED DESIGN DESCRIPTI, pdfauthor=Clinton Jeffery, pdfsubject=, pdfkeywords=}
\usepackage[pdftex]{graphicx}
% Outline numbering
\setcounter{secnumdepth}{5}
\renewcommand\thesection{\arabic{section}}
\renewcommand\thesubsection{\arabic{section}.\arabic{subsection}}
\renewcommand\thesubsubsection{\arabic{section}.\arabic{subsection}.\arabic{subsubsection}}
\renewcommand\theparagraph{\arabic{section}.\arabic{subsection}.\arabic{subsubsection}.\arabic{paragraph}}
\renewcommand\thesubparagraph{\arabic{section}.\arabic{subsection}.\arabic{subsubsection}.\arabic{paragraph}.\arabic{subparagraph}}
\makeatletter
\newcommand\arraybslash{\let\\\@arraycr}
\makeatother
% List styles
\newcommand\liststyleWWviiiNumii{%
\renewcommand\theenumi{\arabic{enumi}}
\renewcommand\theenumii{\arabic{enumii}}
\renewcommand\theenumiii{\arabic{enumiii}}
\renewcommand\theenumiv{\arabic{enumiv}}
\renewcommand\labelenumi{\theenumi)}
\renewcommand\labelenumii{\theenumii.}
\renewcommand\labelenumiii{\theenumiii.}
\renewcommand\labelenumiv{\theenumiv.}
}
% Page layout (geometry)
\setlength\voffset{-1in}
\setlength\hoffset{-1in}
\setlength\topmargin{0.5in}
\setlength\oddsidemargin{1in}
\setlength\evensidemargin{1in}
\setlength\textheight{8.278in}
\setlength\textwidth{6.5in}
\setlength\footskip{0.561in}
\setlength\headheight{0.5in}
\setlength\headsep{0.461in}
% Footnote rule
\setlength{\skip\footins}{0.0469in}
\renewcommand\footnoterule{\vspace*{-0.0071in}\setlength\leftskip{0pt}\setlength\rightskip{0pt plus 1fil}\noindent\textcolor{black}{\rule{0.25\columnwidth}{0.0071in}}\vspace*{0.0398in}}
% Pages styles
\makeatletter
\newcommand\ps@Standard{
  \renewcommand\@oddhead{}
  \renewcommand\@evenhead{\@oddhead}
  \renewcommand\@oddfoot{{\textcolor{black}{\hfill Test Plan Page }}{{\textcolor{black}{\thepage{}}}}}
  \renewcommand\@evenfoot{\@oddfoot}
  \renewcommand\thepage{\arabic{page}}
}
\newcommand\ps@Convertix{
  \renewcommand\@oddhead{}
  \renewcommand\@evenhead{\@oddhead}
  \renewcommand\@oddfoot{}
  \renewcommand\@evenfoot{\@oddfoot}
  \renewcommand\thepage{\arabic{page}}
}
\newcommand\ps@Convertviii{
  \renewcommand\@oddhead{}
  \renewcommand\@evenhead{\@oddhead}
  \renewcommand\@oddfoot{}
  \renewcommand\@evenfoot{\@oddfoot}
  \renewcommand\thepage{\arabic{page}}
}
\newcommand\ps@Convertvii{
  \renewcommand\@oddhead{}
  \renewcommand\@evenhead{\@oddhead}
  \renewcommand\@oddfoot{}
  \renewcommand\@evenfoot{\@oddfoot}
  \renewcommand\thepage{\arabic{page}}
}
\newcommand\ps@Convertvi{
  \renewcommand\@oddhead{}
  \renewcommand\@evenhead{\@oddhead}
  \renewcommand\@oddfoot{}
  \renewcommand\@evenfoot{\@oddfoot}
  \renewcommand\thepage{\arabic{page}}
}
\newcommand\ps@Convertiv{
  \renewcommand\@oddhead{}
  \renewcommand\@evenhead{\@oddhead}
  \renewcommand\@oddfoot{}
  \renewcommand\@evenfoot{\@oddfoot}
  \renewcommand\thepage{\arabic{page}}
}
\newcommand\ps@FirstPage{
  \renewcommand\@oddhead{}
  \renewcommand\@evenhead{\@oddhead}
  \renewcommand\@oddfoot{{\textcolor{black}{\hfill TP Page }}{{\textcolor{black}{\thepage{}}}}}
  \renewcommand\@evenfoot{\@oddfoot}
  \renewcommand\thepage{\arabic{page}}
}
\makeatother
\pagestyle{Standard}
\setlength\tabcolsep{1mm}
\renewcommand\arraystretch{1.3}
\title{TEST PLAN}
\author{Clinton Jeffery}
\date{2011-02-09}
\begin{document}
{\centering TEST PLAN (TP)}

{\centering\selectlanguage{english}\bfseries\color{black}
FOR
\par}


\bigskip

{\centering\selectlanguage{english}\bfseries\color{black}
Project UML (PUMMEL): A UML Diagramming Tool for the 2012 CS 384 Class
\par}


\bigskip


\bigskip


\bigskip

\begin{figure}
\centering
% %\includegraphics[width=3.4354in,height=0.6126in]{TPTemplate-img1.png}
\end{figure}

\bigskip


\bigskip


\bigskip


\bigskip

{\centering\selectlanguage{english}\bfseries\color{black}
Version 1.0
\par}

{\centering\selectlanguage{english}\bfseries\color{black}
March 19, 2012
\par}


\bigskip


\bigskip

{\centering\selectlanguage{english}\bfseries\color{black}
Prepared for:
\par}

{\centering\selectlanguage{english}\bfseries\color{black}
Dr. Clinton Jeffery
\par}


\bigskip


\bigskip

{\centering\selectlanguage{english}\bfseries\color{black}
Prepared by:
\par}

{\centering\selectlanguage{english}\bfseries\color{black}
Coleman Beasley, Alex Dean, Austin Enfield, Jason Fletcher, Cable Johnson, Adrian Norris, Theora Rice, and David Summers 
\par}

{\centering\selectlanguage{english}\bfseries\color{black}
University of Idaho
\par}

{\centering\selectlanguage{english}\bfseries\color{black}
Moscow, ID \ 83844-1010
\par}

{\centering\selectlanguage{english}\bfseries\color{black}
CS384 TPD
\par}

\pagebreak

{\centering\selectlanguage{english}\bfseries\color{black}
RECORD OF CHANGES (Change History)
\par}

\begin{flushleft}
\tablehead{}
\begin{supertabular}{|m{0.5462598in}|m{0.6712598in}|m{1.4212599in}|m{0.23375985in}|m{1.7962599in}|m{0.7337598in}|m{0.6295598in}|}
\hline
~

\centering {\selectlanguage{english}\bfseries\color{black} Change}\par

\centering {\selectlanguage{english}\bfseries\color{black} Number}\par

~
 &
~

\centering \selectlanguage{english}\bfseries\color{black} Date completed
&
~

\centering {\selectlanguage{english}\bfseries\color{black} Location of
change }\par

\centering \selectlanguage{english}\bfseries\color{black} (e.g., page or
figure \#) &
~

\centering {\selectlanguage{english}\bfseries\color{black} A}\par

\centering \selectlanguage{english}\bfseries\color{black} M\newline
D  &
~

\centering {\selectlanguage{english}\bfseries\color{black} Brief
description }\par

\centering \selectlanguage{english}\bfseries\color{black} of change &
~

\centering \selectlanguage{english}\bfseries\color{black} Approved by
(initials) &
~

\centering {\bfseries\color{black} Date }\par

\centering\arraybslash\bfseries\color{black}
approved\\

 &

 &

 &

 &

 &

 &

\\\hline
~
 &
~
 &
~
 &
~
 &
~
 &
~
 &
~
\\\hline
~
 &
~
 &
~
 &
~
 &
~
 &
~
 &
~
\\\hline
~
 &
~
 &
~
 &
~
 &
~
 &
~
 &
~
\\\hline
~
 &
~
 &
~
 &
~
 &
~
 &
~
 &
~
\\\hline
~
 &
~
 &
~
 &
~
 &
~
 &
~
 &
~
\\\hline
~
 &
~
 &
~
 &
~
 &
~
 &
~
 &
~
\\\hline
~
 &
~
 &
~
 &
~
 &
~
 &
~
 &
~
\\\hline
~
 &
~
 &
~
 &
~
 &
~
 &
~
 &
~
\\\hline
~
 &
~
 &
~
 &
~
 &
~
 &
~
 &
~
\\\hline
~
 &
~
 &
~
 &
~
 &
~
 &
~
 &
~
\\\hline
~
 &
~
 &
~
 &
~
 &
~
 &
~
 &
~
\\\hline
~
 &
~
 &
~
 &
~
 &
~
 &
~
 &
~
\\\hline
~
 &
~
 &
~
 &
~
 &
~
 &
~
 &
~
\\\hline
~
 &
~
 &
~
 &
~
 &
~
 &
~
 &
~
\\\hline
~
 &
~
 &
~
 &
~
 &
~
 &
~
 &
~
\\\hline
~
 &
~
 &
~
 &
~
 &
~
 &
~
 &
~
\\\hline
~
 &
~
 &
~
 &
~
 &
~
 &
~
 &
~
\\\hline
~
 &
~
 &
~
 &
~
 &
~
 &
~
 &
~
\\\hline
~
 &
~
 &
~
 &
~
 &
~
 &
~
 &
~
\\\hline
~
 &
~
 &
~
 &
~
 &
~
 &
~
 &
~
\\\hline
~
 &
~
 &
~
 &
~
 &
~
 &
~
 &
~
\\\hline
\end{supertabular}
\end{flushleft}
{\selectlanguage{english}\color{black}
A - ADDED \ M - MODIFIED \ D -- DELETED}

{\centering\selectlanguage{english}\bfseries\color{black}
[ put program /system name here ]
\par}

\pagebreak

{\centering\selectlanguage{english}\bfseries\color{black}
TABLE OF CONTENTS
\par}

{\selectlanguage{english}\bfseries\color{black}
Section\ \ Page}

\setcounter{tocdepth}{9}
\renewcommand\contentsname{}
\tableofcontents

\bigskip

\bigskip
\setcounter{page}{1}\pagestyle{Convertiv}

\section[IDENTIFIER]{\selectlanguage{english}\bfseries\color{black}
TEST PLAN IDENTIFIER}

{\selectlanguage{english}\itshape\color{black}
Some type of unique company generated number to identify this test
plan, its level and the level of software that it is related
to. Preferably the test plan level will be the same as the related
software level. The number may also identify whether the test plan is
a Master plan, a Level plan, an integration plan or whichever plan
level it represents. This is to assist in coordinating software and
testware versions within configuration management.
}

{\selectlanguage{english}\color{black}
[Insert text here.]}


\section[REFERENCES]{\selectlanguage{english}\bfseries\color{black}
REFERENCES}

{\selectlanguage{english}\itshape\color{black}
List all documents that support this test plan. Refer to the actual
version/release number of the document as stored in the configuration
management system. Do not duplicate the text from other documents as
this will reduce the viability of this document and increase the
maintenance effort.
}

{\selectlanguage{english}\color{black}
[Insert text here.]}



\section[INTRODUCTION]{\bfseries\color{black} INTRODUCTION}
{\selectlanguage{english}\color{black}
This test plan is designed to test and evaluate functionality in the Pummel UML editor. It describes a lower level test design. GUI elements, file operations, and diagram error checking will be addressed.
}

\section[TEST ITEMS]{\bfseries\color{black} TEST ITEMS}

{\selectlanguage{english}\itshape\color{black}
These are things you intend to test within the scope of this test
plan. Essentially, something you will test, a list of what is to be
tested. This can be developed from the software application
inventories as well as other sources of documentation and information.
}

{\selectlanguage{english}\color{black}
{\bfseries\color{black} Functions To Be Tested}
\begin{list}{-}{ }
\item Icon Manipulation
\begin{list} {-}{ }
\item Selection (including selecting multiple icons)
\item Insertion
\item Deletion (keyboard, GUI, and right-click)
\item Changing Properties (Inserting/editing text, etc.)
\item Movement
\item Scaling
\item Overlap
\end{list}
\item Line Manipulation
\begin{list} {-}{ }
\item Labelling
\item Drawing
\item Deleting
\item Snapping to Icons
\item Collisions
\end{list}
\item General Editing Tools
\begin{list}{-}{ }
\item Copy (keyboard, GUI)
\item Cut (keyboard, GUI)
\item Paste(Keyboard, GUI)
\item Toggle Grid On/Off
\item Change Insertion Options (Switching between icons and lines to be inserted)
\end{list}
\end{list}

}

\section[SOFTWARE RISK ISSUES]{\bfseries\color{black} SOFTWARE RISK ISSUES}
{\selectlanguage{english}\color{black}
Due to the scholarly nature of this work, it may be hard to update this UML editor or find support for future users. The development team this semester may not be able to provide ongoing support for this product in the years to come. Because of this, those who come to use this program will need to become familiar with the source code in order to fix possible future errors.
}

\section[FEATURES TO BE TESTED]{\bfseries\color{black} FEATURES TO BE TESTED}
{\selectlanguage{english}\itshape\color{black}

This is a listing of what is to be tested from the USERS viewpoint of
what the system does. This is not a technical description of the
software, but a USERS view of the functions.

}
{\selectlanguage{english}\color{black}
\begin{list}{-}{ }
\item High Priority:
\begin{list}{-}{ }
\item Icon Selection (including selecting multiple icons)
\item Icon Deletion (keyboard, GUI, and right-click)
\item Changing Icon Properties
\end{list}
\item Medium Priority:
\begin{list}{-}{ }
\item Change Insertion Options
\item Moving Icons
\item Icon Scaling
\item Drawing Connecting Lines
\item Deleting Connecting Lines
\item Labelling Lines
\end{list}
\item Low Priority:
\begin{list}{-}{ }
\item Toggle Grid On/Off
\item Copy Icon (keyboard, GUI)
\item Cut (keyboard, GUI)
\item Paste(Keyboard, GUI)
\item Undo
\item Redo
\end{list}
\end{list}
}

\section[FEATURES NOT TO BE TESTED]{\bfseries\color{black}
	 FEATURES NOT TO BE TESTED}
{\selectlanguage{english}\itshape\color{black}

This is a listing of what is NOT to be tested from both the Users
viewpoint of what the system does and a configuration
management/version control view. This is not a technical description
of the software, but a USERS view of the functions.

}
{\selectlanguage{english}\color{black}
[Insert text here.]}

\section[APPROACH]{\bfseries\color{black} APPROACH}
{\selectlanguage{english}\itshape\color{black}

This is your overall test strategy for this test plan; it should be
appropriate to the level of the plan (master, acceptance, etc.) and
should be in agreement with all higher and lower levels of
plans. Overall rules and processes should be identified. 

\begin{itemize}
\item Are any special tools to be used? What are they?
\item What metrics will be collected for this test?
\item How many configurations/platforms are to be tested?
\item How will elements in the design deemed "untestable" be processed?
\end{itemize}
}
{\selectlanguage{english}\color{black}
We use three different phases of testing for the entire project:

\begin{enumerate}
\item Unit Testing
\item Integration Testing
\item Functional Testing
\end{enumerate}

\subsection{Unit Testing}
Since the project is almost entirely GUI based, satisfactory unit testing can be a difficult task. Our code allows for a rigorous Model View Control implementation. This was accomplished by separating the logical and GUI based code, as they cannot be tested in the same manner.

Using the saveAsFile method as an example, this method is composed of three parts: invoking a dialogue box to enter the filename, writing the current diagram into XML, and opening a new tab. The first and third parts are invoking GUI events, which means this method cannot be fully unit tested independent of the GUI itself. This means it can only truly be tested during an integration test. However, since writing the XML document contains all the logic of the method (branching, loops, etc.) with no GUI interaction, it can be moved to a helper method and unit tested completely independent of GUI events. The GUI events can be tested in the later integration testing phase. Thus, we can confidently conclude that the saveFileAs process is tested to a satisfactory level.

In order to actually unit test these methods, we will use the CxxTest framework. To write a test for a given class or file, the class and function definitions are written in a header file. From that header file is generated a .cpp file which is compiled individually and linked with the project, producing an executable unit test. Because every building step is automated by an in-house written makefile, the only training needed to use the tool is covered by the also in-house example template that displays proper class syntax and test macro calls.

\subsection{Integration Testing}
NEEDS EDITING
This is the phase where the actual GUI events are tested. During this phase, all of the use cases will be walked through manually while confirming that the right events are being invoked based on input. For these individual use case tests, each test must be run as independently as possible, with minimal set-up. This is to observe the behaviour of each use case completely independent of the others.

\subsection{Functional Testing}
NEEDS EDITING
The functional phase is where the usage of the UML editor is tested to a much higher degree. Rather than testing each use case individually, there will be a variety of users selected to attempt to produce UML diagrams of different types and magnitudes. This will produce a very large variety of permutations of use cases, and allow us to observe how the use cases behave when used together.
}

\section[ITEM PASS/FAIL CRITERIA]{\bfseries\color{black}
	 ITEM PASS/FAIL CRITERIA}
{\selectlanguage{english}\itshape\color{black}
What are the Completion criteria for this plan? This is a critical
aspect of any test plan and should be appropriate to the level of the plan.
}
{\selectlanguage{english}\color{black}
[Insert text here.]}

\section[SUSPENSION CRITERIA]{\bfseries\color{black}
	 SUSPENSION CRITERIA AND RESUMPTION REQUIREMENTS}
{\selectlanguage{english}\itshape\color{black}
If the number or type of defects reaches a point where the follow on
testing has no value, it makes no sense to continue the test; you are
just wasting resources.

Specify what constitutes stoppage for a test or series of tests and
what is the acceptable level of defects that will allow the testing to
proceed past the defects. 
}
{\selectlanguage{english}\color{black}
[Insert text here.]}

\section[TEST DELIVERABLES]{\bfseries\color{black} TEST DELIVERABLES}
{\selectlanguage{english}\itshape\color{black}
What is to be delivered as part of this plan?

\begin{itemize}
\item Test plan document
\item Test cases
\item Relevant error logs or problem reports
\end{itemize}
One thing that is not a test deliverable is the software itself that
is listed under test items and is delivered by development.
}
{\selectlanguage{english}\color{black}
[Insert text here.]}

\section[REMAINING TEST TASKS]{\bfseries\color{black} REMAINING TEST TASKS}
{\selectlanguage{english}\itshape\color{black}
If this is a multi-phase process or if the application is to be
released in increments there may be parts of the application that this
plan does not address. These areas need to be identified to avoid any
confusion should defects be reported back on those future
functions. This will also allow the users and testers to avoid
incomplete functions and prevent waste of resources chasing non-defects.
}
{\selectlanguage{english}\color{black}
[Insert text here.]}

\section[ENVIRONMENTAL NEEDS]{\bfseries\color{black} ENVIRONMENTAL NEEDS}
{\selectlanguage{english}\itshape\color{black}
Are there any special requirements for this test plan, such as:

\begin{itemize}
\item Special hardware such as simulators, static generators etc.
\item How will test data be provided. Are there special collection
	requirements or specific ranges of data that must be provided? 
\end{itemize}

}
{\selectlanguage{english}\color{black}
[Insert text here.]}

\section[STAFFING AND TRAINING NEEDS]{\bfseries\color{black}
	 STAFFING AND TRAINING NEEDS}
{\selectlanguage{english}\itshape\color{black}
Training on the application/system.

Training for any test tools to be used. 
}
{\selectlanguage{english}\color{black}
[Insert text here.]}

\section[RESPONSIBILITIES]{\bfseries\color{black} RESPONSIBILITIES}
{\selectlanguage{english}\itshape\color{black}
Who is in charge?

This issue includes all areas of the plan. Here are some examples:

\begin{itemize}
\item Selecting features to be tested and not tested.
\item Ensuring all required elements are in place for testing. 
\end{itemize}
}
{\selectlanguage{english}\color{black}
[Insert text here.]}

\section[SCHEDULE]{\bfseries\color{black} SCHEDULE}
{\selectlanguage{english}\itshape\color{black}

Should be based on realistic and validated estimates. If the estimates
for the development of the application are inaccurate, the entire
project plan will slip and the testing is part of the overall project plan.

}
{\selectlanguage{english}\color{black}
[Insert text here.]}

\section[PLANNING RISKS AND CONTINGENCIES]{\bfseries\color{black}
	 PLANNING RISKS AND CONTINGENCIES}
{\selectlanguage{english}\itshape\color{black}

What are the overall risks to the project with an emphasis on the
testing process? Specify what will be done for various risk events.

}
{\selectlanguage{english}\color{black}
[Insert text here.]}

\section[APPROVALS]{\bfseries\color{black} APPROVALS}
{\selectlanguage{english}\itshape\color{black}

Who can approve the process as complete and allow the project to
proceed to the next level (depending on the level of the plan)? 

}
{\selectlanguage{english}\color{black}
[Insert text here.]}

\section[GLOSSARY]{\bfseries\color{black} GLOSSARY}

{\selectlanguage{english}\itshape\color{black}

Used to define terms and acronyms used in the document, and testing in
general, to eliminate confusion and promote consistent communications.

}
{\selectlanguage{english}\color{black}
[Insert text here.]}



\clearpage\setcounter{page}{1}\pagestyle{Convertviii}
\section[APPENDIX A. \ [insert name
here{]}]{\selectlanguage{english}\bfseries\color{black} APPENDIX A.
\ [insert name here]}
{\selectlanguage{english}\itshape\color{black}
Include copies of test examples, etc. supplied or
derived from the customer. \ Appendices are labeled A, B, {\dots}n.
\ \ Reference each appendix as appropriate in the text of the document.
}

{\selectlanguage{english}\color{black}
\ [ insert appendix A here ]}

\clearpage\setcounter{page}{1}\pagestyle{Convertix}
\section[APPENDIX B. \ [insert name
here{]}]{\selectlanguage{english}\bfseries\color{black} APPENDIX B.
\ [insert name here]}

\bigskip

{\selectlanguage{english}\color{black}
[ insert appendix B here ]}


\bigskip
\end{document}
